\subsection{Die Aggregatszustände der Materie}

\underbar{Aggregatszustände:} qualitativ unterschiedliche, temperatur- und druckabhängige Erscheinungsformen der Stoffe

\begin{itemize}
	\item Edelgase: keine reguläre Atomordnung
	\item Molekulare Gase: Nahordnung\\
		Moleküle sind regelmäßig und innerlich identisch aufgebaut
	\item Flüssigkeiten: Nahordnung\\
		Ordnung besteht auch hier nur auf Molekularebene
	\item Kristalle: Fernordnung\\
		Ordnung, die sich deutlich weiter als über einen Baustein erstreckt
\end{itemize}

\subsubsection{Gase und Flüssigkeiten}

\begin{center}
	\includegraphics[width=.8\textwidth]{img/2_1_int}
\end{center}

\underbar{Festkörper:}
\begin{itemize}
	\item praktisch inkompressibel
	\item formbeständig
\end{itemize}

\underbar{Flüssigkeit:}
\begin{itemize}
	\item praktisch inkompressibel
	\item leichte Formänderung
	\item haben keine bestimmte Gestalt
	\item haben freie Oberfläche
	\item nimmt die Form des Behälters an, in den sie eingegossen wird
\end{itemize}

\underbar{Gas:}
\begin{itemize}
	\item keine bestimmte Gestalt
	\item hochkompressibel
	\item nehmen immer das ganze freie Volumen des Behälters ein
\end{itemize}

\subsubsection{Kristallstrukturen (ideale Kristalle)}

Kristallstruktur: 3d-Anordnung von Atomen, Molekülen oder Ionen in Kristallen

\subsubsubsection{Kristallgitter}

\begin{center}
	\includegraphics[width=.8\textwidth]{img/2_2_int}
\end{center}

\begin{itemize}
	\item \underbar{Kristallgitter:} abstrakte mathematische Darstellung der Kristallstruktur als eine räumlich periodische Anordnung von \underbar{Punkten} (Punktgitter).
	\item Elementarzelle: das Element des Kristallgitters, mit dem durch \underbar{Translationen} das vollständige Kristallgitter \underbar{lückenlos} reproduziert werden kann.
	\item Gitterkonstanten: die Beträge der Kanten und Winkel der Elementarzelle
	\item Kristallsysteme\\
		\begin{center}
			\includegraphics[width=.8\textwidth]{img/2_3_int}
		\end{center}
		Wichtig für uns: hexagonal, kubisch-primitv, kubisch-raumzentriert und kubisch-flächenzentriert
	\item Bravais Gitter
	\item Atome in der Elementarzelle
		\begin{itemize}
			\item Zahl von Atomen in der Elementarzelle: $N_{EZ}$\\
				$N_{EZ}$ krz: 2; $N_{EZ}$ kfz: 4
			\item Packungsdichte (PD)\\
				$$\frac{\textrm{Atomvolumen in der EZ}}{\textrm{Volumen der EZ}}$$
			\item Koordinationszahl (KZ)\\
				Anzahl nächster Nachbarn jedes Atoms
			\item Miller'sche Indizes
				\begin{itemize}
					\item Ebenen\\
						\begin{center}
							\includegraphics[width=.8\textwidth]{img/2_4_int}
						\end{center}
					\item Richtungen\\
						in kubischen Gittern: $[100] \perp (100)$; $[110] \perp (110)$; $[111] \perp (111)$
						\begin{center}
							\includegraphics[width=.8\textwidth]{img/2_5_int}
						\end{center}
				\end{itemize}
		\end{itemize}
\end{itemize}

\subsubsubsection{Metalle}

\underbar{Kubisch primitives Gitter}

\begin{itemize}
	\item $N_{EZ}$: 1 (cP1)
	\item KZ: 6
	\item PD: 52\%
	\item sehr selten (z.B. Polonium)
\end{itemize}

\begin{minipage}{.4\textwidth}
	\underbar{Kubisch raumzentriertes Gitter}
	\begin{itemize}
		\item $N_{EZ}$: 2 (cI2)
		\item KZ: 8
		\item PD: 68\%
		\item harte Metalle: $\alpha$-Fe, Cr, W, Mo
	\end{itemize}	
\end{minipage}
\begin{minipage}{.6\textwidth}
	\includegraphics[width=\textwidth]{img/2_6_int}
\end{minipage}

\begin{minipage}{.4\textwidth}
	\underbar{Kubisch flächenzentriertes Gitter}
	\begin{itemize}
		\item $N_{EZ}$: 4 (cF4)
		\item KZ: 12
		\item PD: 74\%
		\item weiche Metalle: Au, Ag, Cu, Al, Ni, Pb
	\end{itemize}	
\end{minipage}
\begin{minipage}{.6\textwidth}
	\includegraphics[width=\textwidth]{img/2_7_int}
\end{minipage}

\begin{minipage}{.4\textwidth}
	\underbar{Hexagonal dichteste Packung}
	\begin{itemize}
		\item $N_{EZ}$: 2 (hP2)
		\item KZ: 12
		\item PD: 74\%
		\item Zn, Cd, Hg
	\end{itemize}	
\end{minipage}
\begin{minipage}{.6\textwidth}
	\includegraphics[width=\textwidth]{img/2_8_int}
\end{minipage}

\newpage
\subsubsubsection{Kovalente Kristalle}

\includegraphics[width=\textwidth]{img/2_9_int}

\begin{itemize}
	\item $N_{EZ}$: 8 (cF8)
	\item KZ: 4
	\item PD: 34\%
	\item Diamant, Silizium, Ge
\end{itemize}








