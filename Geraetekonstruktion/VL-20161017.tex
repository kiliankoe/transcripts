\underbar{b) Allgemeines Gerätemodell}

\begin{minipage}{.3\textwidth}
	Betrachtungsebene\\ funktioneller Struktur
	\begin{itemize}
		\item \underbar{Kommunikationsebene}\\
			Stoff-, Energie- und Informationsaustausch zwischen Mensch und Gerät oder Gerät und Gerät
		\item \underbar{Verarbeitungsebene}\\
			Art und Weise der definierten Verarbeitung von Eingangsgrößen zu Ausgangsgrößen
		\item \underbar{Störgrößenebene}\\
			Maßnahmen zur Sicherung der Gerätefunktion und zum Schutz gegen Einwirkungen von außen und nach außen
	\end{itemize}
\end{minipage}
\begin{minipage}{.3\textwidth}
	\includegraphics[width=\textwidth]{img/2_1}
\end{minipage}
\begin{minipage}{.3\textwidth}
	Betrachtungsebene geometrisch-stofflicher\\ Struktur
	\begin{itemize}
		\item Bedien- und Anzeigeelemente\\ (z.B. Tasten, Monitor)
		\item Interfaceelemente\\ (z.B. Stecker, Buchsen, mechanische Verbindungen)
		\item Bauelemente zur Realisierung der Verarbeitungsfunktion\\ (z.B. IC (Integrierter Schaltkreis), Transistor, Welle)
		\item Bauelemente zur Stützfunktion\\ (z.B. Gestell, Träger)
		\item Bauelemente zur Schutzfunktion\\ (z.B. Gehäuse, Umhüllung, Abschirmung, Wärmeabführung)
	\end{itemize}
\end{minipage}

\Ra vgl. erweitertes allgemeines Gerätemodell mit Softwareanteil <Folie 36>

\subsection{Hierarchieebenen der Gerätegestaltung}

\underbar{a) Hierarchieebenen der Funktionselemente}

\begin{tabular}{p{.3\textwidth} p{.3\textwidth} p{.3\textwidth}}
\hline
Funktionsschicht/\newline Wirkfläche & sind die am Funktionsfluss direkt beteiligten Oberflächen, Geometrien, Schichten mit entsprechenden werkstofflichen Eigenschaften & z.B. Widerstandsschicht\\ \hline
Einzelteil & niedrigste Ebene der körperlichen Zerlegung eines Geräts, bestimmt durch Werkstoff und Geometrie & z.B. Zahnrad, Widerstandsbauelement\\ \hline
Baugruppe & abgegrenzte, selbständige Gruppe von Einzelteilen, die miteinander verkoppelt sind und eine prüfbare Teilfunktion realisieren & z.B. Leiterplattenbaugruppe, Getriebe\\ \hline
\end{tabular}

\begin{tabular}{p{.3\textwidth} p{.3\textwidth} p{.3\textwidth}}
\hline
Einzelgerät & aus Baugruppen und Einzelteilen zusammengesetztes System mit einer Gesamtfunktion & z.B. Zählwerk, Messgerät, Netzteil\\ \hline
Anlage & Erfüllung komplexer, mehrerer Funktionen durch Verknüpfung von mehreren Geräten & z.B. Messplatz, Automatisierungsanlage\\ \hline
\end{tabular}

\vspace{1cm}
\underbar{\textbf{Beachte:}}

\begin{tabular}{p{.3\textwidth} p{.3\textwidth} p{.3\textwidth}}
\hline
Bauelement & werden bei der Betrachtung nicht weiter zerlegt und können Einzelteil oder Baugruppe sein & z.B. Widerstandsbauelement, gehäustes Halbleiterbauelement\\ \hline
\end{tabular}

\vspace{1cm}
\underbar{b) Hierarchieebenen der Verbindungen und Verdrahtungen}

Nutzung von Verbindungen und Verdrahtungen (elektronische und elektrische Verbindungen) zur Realisierung der nächsthöheren Hierarchieebene der Funktionselemente.

z.B.

\begin{itemize}
	\item interne Bauelementeverdrahtung
	\item Verbindung auf dem Verdrahtungsträger zur elektronischen Baugruppe
	\item interne Geräteverdrahtung zwischen den Baugruppen
	\item Geräteverdrahtung als externe Verbindung zwischen Geräten zu einer Anlage
\end{itemize}